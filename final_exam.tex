\documentclass[12pt, a4paper]{article}
\usepackage{graphicx}
\graphicspath{ {./imgs/} }
\usepackage{float}
\usepackage{listings}
\usepackage{color}

\definecolor{codegreen}{rgb}{0,0.6,0}
\definecolor{codegray}{rgb}{0.5,0.5,0.5}
\definecolor{codeorange}{rgb}{1,0.49,0}
\definecolor{backcolour}{rgb}{0.95,0.95,0.96}

\lstset{
  backgroundcolor=\color{backcolour},   
  commentstyle=\color{codegray},
  keywordstyle=\color{codeorange},
  numberstyle=\tiny\color{codegray},
  stringstyle=\color{codegreen},
  basicstyle=\ttfamily\footnotesize,
  breakatwhitespace=false,         
  breaklines=true,                 
  captionpos=b,                    
  keepspaces=true,                 
  numbers=left,                    
  numbersep=5pt,                  
  showspaces=false,                
  showstringspaces=false,
  showtabs=false,                  
  tabsize=2,
  xleftmargin=10pt,
}
\renewcommand{\lstlistingname}{Code}

\counterwithin{figure}{section}

%to write code:
%\begin{lstlisting}[language=java, caption={my caption}]
%    
%\end{lstlisting}

%to insert an image
%\begin{figure}[H]
%  \centering
%  \includegraphics[width=\columnwidth]{img.png}
%  \caption{description of the image}
%\end{figure}

%\title{\large{About Server Side Public license}\\Open Source software and profiting from them}
\title{%
  Server Side Public License \\
  \large Cloud providers and the future of open source
}
\author{Marrocco Simone}

\setcounter{section}{-1}

\begin{document}
  \maketitle
  \begin{figure}[H]
    \centering
    \includegraphics[scale=0.35]{unitn.png}
  \end{figure}
  \tableofcontents

  \section{Acknowledgments}
  The idea behind this document came from the video "Fork you ElasticSearch! How Open Source Works" \footnote{https://youtu.be/tzq4asJegKY} from the "Fireship" youtube channel, where he talked about the recent (for that time) change of license of the popular open source software Elasticsearch, what it means for the open source community and most importantly what open source is. 

  In the pages that follow we are expanding the concepts behind this video, going deeper into the problems that companies that have an open source business model face, discussing what open source is, and how to balance the idea of free software with the necessities of protecting free software developers from tech giants that use their programs to make millions but give back nothing in return.

  \pagebreak
  \section{History of the Server Side Public license}
  \subsection{Open source software abuse by cloud companies}
  Millions of people work every day on open source software. People volunteer their time to create or improve digital tools that everyone can use. This is done in a way that allows everyone to look at your code so that it can be free: the idea is that you, as a user, should have total control over the program, and not the other way around.

  Giving the software you created for free does not mean that you cannot build a business model upon it. There are a lot of companies that managed to earn from offering consulting or their software as a service. For example, database creators may offer a cloud service with bonuses, like scalability and automatic backups, to help other businesses store their application data without the need for a complex initial setup.

  But this can become difficult when a big company providing cloud services, like Google or Amazon, offers the same type of service, gaining from the fact that the software itself is free. To make things worse, the creators may use the same cloud storage services to save their data, effectively being both competitors and partners and making it difficult to compete against them. When dealing with the same product, clients may prefer the bigger company option just because of its brand or because they aggregate multiple products altogether. 
  \footnote{https://siliconangle.com/2018/12/22/stormy-weather-open-source-software-firms-limit-licenses-stop-cloud-giants/} 
  \footnote{https://www.forbes.com/sites/glennsolomon/2020/10/13/finding-trillions-in-the-clouds-how-open-source-companies-can-win-in-a-world-dominated-by-amazon-microsoft-and-google/}

  Normally, this would not be a problem: while being a competitor, the core of Open Source is that every user contributes to the product and makes it better. But the "Cloud Giants" do not do that, preferring to just use it without helping the original creator, either by improving the software or contributing economically. "[Amazon has] taken Redis' open source project and packaged it as a service and monetized that,” Manish Gupta, CMO of Redis Labs, told Business Insider. "Their contribution to the open source is minuscule, not even in the 1 percent range.” 
  \footnote{https://www.businessinsider.com/mongodb-and-redis-labs-have-a-new-plan-to-take-on-amazon-2018-10}

  % * https://www.businessinsider.com/mongodb-and-redis-labs-have-a-new-plan-to-take-on-amazon-2018-10
  % %Every year, large cloud companies like Amazon rake in billions of dollars — but some of their most popular cloud services comes from repackaging software projects created by other, smaller companies. Amazon is repackaging what's known as "open source" software, which is software that is given away for free, meaning Amazon has every legal right to use it in this way. For instance, since 2013, Amazon had been offering the open source database Redis as part of a popular cloud service called ElastiCache. While most open source licenses allow anyone to use the software for free, the users typically contribute back to the software project: fixing things, adding features, making the software better for everyone. 
  % Although those major cloud companies technically aren't breaking any laws, if they are all take and no give, they are breaking the social rules of open source. Companies like MongoDB and Redis Labs can give away big portions of their software because what they get in return is free labor from the software users to make the software better. That helps them improve the commercial versions of their software, the ones they do charge money to use. There's a term in the open source community for what the cloud companies are being accused of doing: strip mining. It means that companies are using free software code for their own gain, with no intention of contributing to the project.

  % * https://siliconangle.com/2018/12/22/stormy-weather-open-source-software-firms-limit-licenses-stop-cloud-giants/
  %It's the latest development in the ongoing struggle by open source developers to come up with sustainable business models built upon software that is essentially free. Open source has transformed the software industry, but only a few companies such as Red Hat Inc. — itself likely to be acquired by IBM Corp. in a recently announced deal — are consistently profitable.
  % Amazon's tactics are "the worst behavior I've ever seen in the software industry and it's all because of a loophole in the licensing,” said Michael Howard, chief executive of MariaDB Corp., which sells software built upon an open source base. Howard charged that the cloud giant is "strip mining” by exploiting the work of a community of developers who work for free. For its part, Amazon said it complies fully with all license terms. "If a company wants the exclusive right to build a business around the source code, they should make the source code closed and proprietary,” a spokesman said. "Open source software should remain open and free of licenses that make it more encumbered.” 
  % Cloud computing companies don't technically distribute code. Their creations are consumed as a service and code never changes hands. That exempts them from the need to share changes back to the community, effectively giving them the ability to gain a proprietary advantage from the work of others. For example, Amazon's Aurora MySQL is based upon the open source MySQL database management system covered by the GNU Public License.

  % * https://www.forbes.com/sites/glennsolomon/2020/10/13/finding-trillions-in-the-clouds-how-open-source-companies-can-win-in-a-world-dominated-by-amazon-microsoft-and-google/
  % the big cloud service providers—Amazon Web Services (AWS), Microsoft Azure, and Google Cloud Platform—are also eager to generate revenue from open source code. They often offer low-cost hosted versions of popular open source software, including MongoDB and Elastic to directly compete with commercial open source companies. For example, Amazon offers a product called DocumentDB (with MongoDB compatibility) priced sharply relative to MongoDB's own cloud-hosted version of its unstructured database, called Atlas. MongoDB and other large open source companies argue their cloud offerings are more robust, have deeper connectivity to third-party applications, and come with expert service and support. All of that may be true, but that hasn't stopped AWS from finding ways to muscle in on the open source gold rush. The cloud vendors also offer the promise of scale, and the performance and availability that they've become famous for, making them very formidable foes.
  \subsection{Color.Js and Faker.Js}
  Some programmers, discouraged by these practices, decide to abandon their open source projects. Marak was the developer of two popular libraries, Color.js and Faker.js, for the NodeJs environment. The Color library received over 20 million weekly downloads on npm alone and had almost 19,000 projects relying on it. Whereas, Faker received over 2.8 million weekly downloads on npm, and had over 2,500 dependents. On January 9, 2022, he committed a new version of both libraries, deleting the entire code and replacing it with lines that broke every software that used them. \footnote{https://infotechlead.com/software/github-developer-marak-squires-corrupts-open-\\source-libraries-70616}

  The creator of those libraries, expressing feelings of dissatisfaction, posted on Github: "Respectfully, I am no longer going to support Fortune 500s ( and other smaller-sized companies ) with my free work. There isn't much else to say. Take this as an opportunity to send me a six-figure yearly contract or fork the project and have someone else work on it" \footnote{https://web.archive.org/web/20210704022108/https://github.com/Marak/faker.js\\/issues/1046}. Even though the open source ideology is not to make money, volunteers are not happy working freely, only to then have other businesses use the library without contribution. Various people commented on Marak's post, encouraging his actions. 

  % * https://infotechlead.com/software/github-developer-marak-squires-corrupts-open-source-libraries-70616 
  % On January 9, 2022 Users of popular open source libraries 'colors' and 'faker' were left stunned after they saw their applications, using these libraries, printing gibberish data and breaking. Some surmised if the NPM libraries had been compromised, but it turns out there's much more to the story. The developer of these libraries intentionally introduced an infinite loop that bricked thousands of projects that depend on 'colors' and 'faker.' The colors library receives over 20 million weekly downloads on npm alone and has almost 19,000 projects relying on it. Whereas, faker receives over 2.8 million weekly downloads on npm, and has over 2,500 dependents.
  % In November 2020, Marak had warned that he will no longer be supporting the big corporations with his "free work" and that commercial entities should consider either forking the projects or compensating the dev with a yearly "six figure" salary. "Respectfully, I am no longer going to support Fortune 500s ( and other smaller sized companies ) with my free work. There isn't much else to say," the developer previously wrote. "Take this as an opportunity to send me a six figure yearly contract or fork the project and have someone else work on it.
  % Marak's bold move has opened up a can of worms and attracted mixed responses. Some members of the open source software community have praised the developer's actions, while others are appalled by it. Concerns emerged as to how big businesses were used to "exploiting" open source; by consuming it incessantly but not giving back enough to support the unpaid volunteers who sustain these critical projects by giving up their free time.
  % "The responses to the colors.js/faker.js author sabotaging their own packages are really telling about how many corporate developers think they are morally entitled to open source developers' unpaid labour without contributing anything back," wrote one Twitter user.
  \subsection{MongoDB}
  Companies that started as open source projects and invested heavily try to fight back against Cloud Giants becoming competitors without contributing. And they usually do that by restricting their license.

  MongoDB is one of the companies that fought against Amazon by creating a new license, the Server Side Public License, for its software new version. Built on top of the General Public License v3, it aims to discourage other people from offering its software as a cloud service unless they are willing to share the specifications and the code of the entire infrastructure used, in a strong copyleft way. We will discuss later in detail what this license means for the Open Source community.

  On their website, MongoDB said about the change: "The market is quickly moving to consume most software as a service. This is a time of incredible opportunity for open source projects, with the potential to foster a new wave of great open source server-side software. The reality, however, is that once an open source project becomes interesting, it is too easy for large cloud vendors to capture all the value but contribute nothing back to the community. The community needs a new license that builds on the spirit of the AGPL, but makes explicit the conditions for providing the software as a service." \footnote{https://www.mongodb.com/licensing/server-side-public-license/faq}

  However, things did not go as planned for the company. Debian, Red Hat Enterprise Linux, and Fedora later dropped MongoDB, after deeming the license problematic. Furthermore, the change did not stop Amazon from offering a competing service: the AWS team forked the previous version, licensed under the Affero GPL, and created DocumentDB, a similar product fully compatible with its predecessor \footnote{https://en.wikipedia.org/wiki/Server\_Side\_Public\_License}: "Given that DocumentDB is designed to work with a version of MongoDB released before that license went into effect, the SSPL doesn't appear to apply to DocumentDB. But AWS thinks that it will be able to help companies that have tried and struggled to implement MongoDB on their own achieve better performance and scale than MongoDB (the corporation) can provide." 
  \footnote{https://www.geekwire.com/2019/amazon-web-services-calls-mongodbs-licensing-\\bluff-documentdb-new-managed-database/}

  % * https://www.mongodb.com/licensing/server-side-public-license/faq
  % The market is quickly moving to consume most software as a service. This is a time of incredible opportunity for open source projects, with the potential to foster a new wave of great open source server side software. The reality, however, is that once an open source project becomes interesting, it is too easy for large cloud vendors to capture all the value but contribute nothing back to the community. The community needs a new license that builds on the spirit of the AGPL, but makes explicit the conditions for providing the software as a service. Rather than litigating this issue in the courts, we are issuing a new license to eliminate any confusion about the specific conditions of offering a publicly available MongoDB as a service. This change is also designed to make sure that companies who do run a publicly available MongoDB as a service, or any software subject to the SSPL, are giving back to the community. It should be noted that the new license maintains all of the same freedoms the community has always had with MongoDB under AGPL - they are free to use, review, modify, and redistribute the source code. The only changes are additional terms that make explicit the conditions for offering a publicly available MongoDB as a service.

  % * https://en.wikipedia.org/wiki/Server_Side_Public_License
  % In October 2018, the MongoDB database was relicensed under the SSPL. Debian, Red Hat Enterprise Linux, and Fedora subsequently dropped MongoDB, citing concerns about the SSPL. Amazon released a partially compatible but proprietary service named DocumentDB.

  % * https://www.geekwire.com/2019/amazon-web-services-calls-mongodbs-licensing-bluff-documentdb-new-managed-database/
  % Given that DocumentDB is designed to work with a version of MongoDB released before that license went into effect, the SSPL doesn't appear to apply to DocumentDB. But AWS thinks that it will be able to help companies that have tried and struggled to implement MongoDB on their own achieve better performance and scale than MongoDB (the corporation) can provide.
  \subsection{Other companies following up}
  While the Open Source community discussed the problems that pose both this new type of license and the power of the cloud providers, other companies followed the example of MongoDB. 

  \subsubsection{Elastic}
  Elastic, the owner of the two popular tools Elasticsearch and Kibana, decided to change its license from Apache 2.0 to a dual license, one of which is SSPL. The user can choose between SSPL and a proprietary license which allows the use of the software freely but prohibits hosting the software as a service without Elastic consent. In their blog, the company wrote: "So why the change? AWS and Amazon Elasticsearch Service. They have been doing things that we think are just NOT OK since 2015 and it has only gotten worse. If we don't stand up to them now, as a successful company and leader in the market, who will? We think that Amazon's behavior is inconsistent with the norms and values that are especially important in the open source ecosystem. Our hope is to take our presence in the market and use it to stand up to this now so others don't face these same issues in the future." \footnote{https://www.elastic.co/blog/why-license-change-aws}, "We do not have a commercial relationship with AWS on the Amazon Elasticsearch Service. We do not actively support that service, and no longer want our investments in our software to directly benefit that service. For transparency, we also have ongoing litigation with AWS" \footnote{https://www.elastic.co/pricing/faq/licensing}

  % * https://www.elastic.co/blog/why-license-change-aws
  % This was an incredibly hard decision, especially with my background and history around Open Source. I take our responsibility very seriously. And to be clear, this change most likely has zero effect on you, our users. It has no effect on our customers that engage with us either in cloud or on premises. Its goal, hopefully, is pretty clear. So why the change? AWS and Amazon Elasticsearch Service. They have been doing things that we think are just NOT OK since 2015 and it has only gotten worse. If we don't stand up to them now, as a successful company and leader in the market, who will? 
  % We think that Amazon's behavior is inconsistent with the norms and values that are especially important in the open source ecosystem. Our hope is to take our presence in the market and use it to stand up to this now so others don't face these same issues in the future. In the open source world, trademarks are considered a great and positive way to protect product reputation. Trademarks have been used and enforced broadly. They are considered sacred by the open source community, from small projects to foundations like Apache to companies like RedHat. So imagine our surprise when Amazon launched their service in 2015 based on Elasticsearch and called it Amazon Elasticsearch Service. We consider this to be a pretty obvious trademark violation. NOT OK. We have seen that this trademark issue drives confusion with users thinking Amazon Elasticsearch Service is actually a service provided jointly with Elastic, with our blessing and collaboration. This is just not true. NOT OK. When the service launched, imagine our surprise when the Amazon CTO tweeted that the service was released in collaboration with us. It was not. And over the years, we have heard repeatedly that this confusion persists. NOT OK. 

  % * https://www.elastic.co/pricing/faq/licensing
  %We are moving our Apache 2.0-licensed source code in Elasticsearch and Kibana to be dual licensed under the Elastic License and Server Side Public License (SSPL), giving users the choice of which license to apply. We are also simplifying the Elastic License (Elastic License v2, or ELv2) and making it substantially more permissive. This license change ensures our community and customers have free and open access to use, modify, redistribute, and collaborate on the code. It also protects our continued investment in developing products that we distribute for free and in the open by restricting cloud service providers from offering Elasticsearch and Kibana as a service without contributing back. This will apply to all maintained branches of these two products starting with the 7.11 release. Our default distribution will continue to be under the Elastic License. 
  % As mentioned above, broadly we are aiming to collaborate with public cloud providers that take our products and provide them as a service. We have built strong relationships with Google Cloud, Microsoft Azure, Alibaba Cloud, and Tencent Cloud. We also partner with AWS with our listing of Elastic Cloud in the AWS Marketplace, and continue to invest in that relationship to make Elastic Cloud the best hosted Elasticsearch and Kibana experience on AWS. However, we do not have a commercial relationship with AWS on the Amazon Elasticsearch Service. We do not actively support that service, and no longer want our investments in our software to directly benefit that service.For transparency, we also have ongoing litigation with AWS, discussed here and here. 

  \subsubsection{Redis}
  Redis is another popular database that focuses on the velocity of retrieving information. Its main strength comes from six modules that allow Redis to be used as a SQL, NOSQL or other popular database structure choices. The company working on the project was one of the first to fight against cloud vendors. In August 2018 the CTO Yiftach Shoolman wrote how "Cloud providers have been taking advantage of the open source community for years by selling (for hundreds of millions of dollars) cloud services based on open source code they didn't develop (e.g. Docker, Spark, Hadoop, Redis, Elasticsearch and others). This discourages the community from investing in developing open source code" \footnote{https://redis.com/blog/redis-license-bsd-will-remain-bsd/} and decided to change the license from Apache 2.0 to one called Common Close License, which is more restrictive. In February 2019, to clear the confusion on this license, Redis modules changed to a proprietary one called Redis Source Available License (RSAL). "Over time, other respected open source companies, like MongoDB and Confluent, created their own proposals for modern variants to open source licensing. Each company took a different approach, but all shared the same goal — stopping cloud providers from taking successful open source projects that were developed by others, packaging them into proprietary services, and using their market power to generate significant revenue streams" \footnote{https://redis.com/blog/redis-labs-modules-license-changes/}. In November 2022 they followed Elastic by double licensing their modules under both the second version of RSAL and SSPL, calling the MongoDB one "the de facto standard for source available licenses". "RSALv2 is a permissive non-copyleft license, allowing the right to "use, copy, distribute, make available, and prepare derivative works of the software” and has only two primary limitations. Under RSALv2, you may not: Commercialize the software or provide it to others as a managed service;  Remove or obscure any licensing, copyright, or other notices." \footnote{https://redis.com/blog/rsalv2-sspl-announcement/}

  % * https://redis.com/blog/redis-license-bsd-will-remain-bsd/
  % (August 22, 2018) Cloud providers have been taking advantage of the open source community for years by selling (for hundreds of millions of dollars) cloud services based on open source code they didn't develop (e.g. Docker, Spark, Hadoop, Redis, Elasticsearch and others). This discourages the community from investing in developing open source code, because any potential benefit goes to cloud providers rather than the code developer or their sponsor. This new license allows full use of our Redis modules under the popular, liberal Apache v2.0 terms, but restricts the selling of the modules themselves. That means you can build internal, external and commercial products on top of our modules and sell those, but cannot directly sell the original modules. We believe this licensing supports the open and free use of modules, while still maintaining our rights over commercializing our assets.

  % * https://redis.com/blog/redis-labs-modules-license-changes/
  % (February 21, 2019) Early in August 2018, Redis was one of the first open source companies to realize that the current open source licensing scheme falls short when it comes to use by cloud providers. We wanted to make sure open source companies could continue to contribute to their projects, while still maintaining sustainable business in the cloud era. That's why we changed the license of our Redis Modules from AGPL to Apache2 modified with Commons Clause. However, over time, other respected open source companies, like MongoDB and Confluent, created their own proposals for modern variants to open source licensing. Each company took a different approach, but all shared the same goal — stopping cloud providers from taking successful open source projects that were developed by others, packaging them into proprietary services, and using their market power to generate significant revenue streams. After many discussions with members of our community, we decided to change the license of our Redis Modules to Redis Source Available License (RSAL).
  % RSAL is a software license created by Redis for certain Redis Modules running on top of open source Redis. RSAL grants equivalent rights to permissive open source licenses for the vast majority of users. With RSAL, developers can use the software; modify the source code' integrate it with an application; and use, distribute or sell their application. The only restriction is that the application cannot be a database, a caching engine, a stream processing engine, a search engine, an indexing engine or an ML/DL/AI serving engine.

  % * https://redis.com/blog/rsalv2-sspl-announcement/
  % (November 15, 2022) The new RSALv2 license is simple to read and makes its permissions and limitations clear. And, for those users who require a more standardized license, we hope that the added option to use our software under the SSPL opens Redis Stack and our Redis modules to an even wider audience. Created by MongoDB and adopted by Elastic and many others, SSPL is becoming the de facto standard for source available licenses and is being used by millions of developers worldwide.
  % RSALv2 is a permissive non-copyleft license, allowing the right to "use, copy, distribute, make available, and prepare derivative works of the software” and has only two primary limitations. Under RSALv2, you may not: Commercialize the software or provide it to others as a managed service;  Remove or obscure any licensing, copyright, or other notices
  % We believe that the permissive approach of RSALv2 and the standard wording we use to define its limitations solve many of the challenges raised by our community, but we are also aware that, like any newly created license, it will take time for some users (and their legal teams) to digest it. For this reason, we've also added an option to use the SSPL. This dual-license approach will allow users to choose between a permissive but less well-known license, RSALv2, or a more standardized but copyleft license, such as SSPL.

  \pagebreak
  \subsubsection{Confluent}
  Some companies decided to block entirely the possibility of cloud giants abusing their software. Confluent Inc. is the primary developer of a managing system for Apache Kafka, a popular open source tool used to pipeline an enormous stream of data offered under Apache 2.0. Confluent main work is on components built upon Kafka to improve it. After Amazon announced that it was now offering the same software developed by Confluent as a competing service, the company changed its license to the Confluent Community License, which mostly allows the same liberties as an Open Source one, but prohibits offering the software as a service 
  \footnote{https://siliconangle.com/2018/12/22/stormy-weather-open-source-software-firms-\\limit-licenses-stop-cloud-giants/}.
  "What this means is that, for example, you can use KSQL however you see fit as an ingredient in your products or services, whether those products are delivered as software or as SaaS, but you cannot create a KSQL-as-a-service offering" \footnote{https://www.confluent.io/blog/license-changes-confluent-platform/}. In the same blog post explaining why the change, the CEO Jay Kreps writes, to justify his decision, that "the cloud providers have significant advantages: they control the pricing of all resources a service provider will use and can tightly integrate their own services across all their offerings. [...] [Some of those] take the open source code, bake it into the cloud offering, and put all their own investments into differentiated proprietary offerings". The co-founder then explains how this move was not done for greed or a financial problem of the company, as the project was not started to make money yet it is doing exceedingly well.

  % * https://www.confluent.io/blog/license-changes-confluent-platform/
  % We're changing the license for some of the components of Confluent Platform from Apache 2.0 to the Confluent Community License. This new license allows you to freely download, modify, and redistribute the code (very much like Apache 2.0 does), but it does not allow you to provide the software as a SaaS offering (e.g. KSQL-as-a-service). What this means is that, for example, you can use KSQL however you see fit as an ingredient in your own products or services, whether those products are delivered as software or as SaaS, but you cannot create a KSQL-as-a-service offering. We'll still be doing all development out in the open and accepting pull requests and feature suggestions. For those who aren't commercial cloud providers, i.e. 99.9999% of the users of these projects, this adds no meaningful restriction on what they can do with the software, while allowing us to continue investing heavily in its creation.
  % First, is this kind of investment necessary at all? For many simple open source projects, I don't think it is. There are thousands of libraries thriving on GitHub that don't need much investment beyond a few volunteer contributors. Distributed data systems are different. Building a successful new distributed data platform is just excruciatingly hard. You don't need to take my word for it, though, it turns out this experiment has been done. Dozens of NoSQL databases emerged in the 2009-2010 timeframe. Some were created as part-time projects, some came out of the internal infrastructure of large web companies, and some were created as commercial ventures. What I think is most stark is that the only systems that remained relevant through to today are those that, whatever their origin, managed to develop a stable commercial entity that helped sustain ongoing investment. Those that did this (MongoDB, ElasticSearch, Cassandra, Hadoop) all continue to thrive and have become part of the modern stack. Those that didn't (Voldemort, Dynomite, CouchDB, and a dozen others) have all fallen by the wayside, despite early popularity. They still exist, but most likely you have never heard of them. In other words I do believe business can help fund a virtuous cycle of open source contribution. In a world in which data systems are delivered as on-premises software, we as an industry have figured out how to build sustainable companies that can drive this kind of virtuous cycle. It isn't easy, but starting a company never is. In that model, we've found that permissive open source licensing such as Apache 2.0 can be the major component of a thriving software offering that sustains a healthy business. However, the world has significantly changed with the rise of cloud offerings that provide this kind of software as a service. In this new world, the cloud providers have significant advantages: they control the pricing of all resources a service provider will use and can tightly integrate their own services across all their offerings.  The major cloud providers (Amazon, Microsoft, Alibaba, and Google) all differ in how they approach open source. Some of these companies partner with the open source companies that offer hosted versions of their system as a service. Others take the open source code, bake it into the cloud offering, and put all their own investments into differentiated proprietary offerings. The point is not to moralize about this behavior, these companies are simply following their commercial interests and acting within the bounds of what the license of the software allows. As a company, one solution we could pursue would be for us to build more proprietary software and pull back from our open source investments. But we think the right way to build fundamental infrastructure layers is with open code. As workloads move to the cloud we need a mechanism for preserving that freedom while also enabling a cycle of investment, and this is our motivation for the licensing change.
  % There are two cynical interpretations that I worry about with this announcement. The first is that this indicates that Confluent is struggling and needs to do this to make money. This is not the case, Confluent is doing exceedingly well, and we think that is a fantastic thing, both for our customers, and for our ability to invest in the community and open source. Our goal with this change is to ensure that we are able to sustain that growth, and continue to invest in open and free offerings. The second cynical interpretation is, ironically, the opposite: that this is part of a greedy ploy to extract ever more money by a rapacious corporation. Against that I can say only this: Confluent was not created as just a way to make money. We have a vision for the architecture of a modern data-driven company that is centered around streams of events. We want to make that a reality in the world. Confluent is a group of people that believe in this idea, and for many of us, our work on this project predates Confluent itself. That early work in building code and community wasn't part of a decade long plot to commercialize it. Rather, we think this plan to re-architect all the companies in the world around event streams is a bold one and will take a lot of work. This change puts us in a position to continue that work for the decades ahead and contribute to the software, community, and practices that make it a reality.

  % * https://siliconangle.com/2018/12/22/stormy-weather-open-source-software-firms-limit-licenses-stop-cloud-giants/
  % Confluent Inc. stoked the fire last week by announcing license changes to portions of its product line that explicitly prohibit cloud providers from delivering its software as a service. Confluent is the principal developer of the popular Kafka stream processing platform. The company made the move in response to AWS' announcement last month that it will sell a managed Kafka service. On the one hand, building businesses on top of open source products generates revenue that can be funneled back into further development. That's the case Confluent co-founder and CEO Jay Kreps made in a blog post announcing the license change. "Many of the people who were eking out small contributions late at night on a pure-passion basis could now get paid to contribute full-time,” he wrote. In raising more than $80 million toward the goal of building a profitable company, he added, "Confluent could fund not just code contributions but also the sizable cloud bill to run the kind of large scale distributed torture tests that are needed to keep a code base stable while scaling contribution from a growing community.”
  
  \pagebreak
  \section{Is SSPL an open source license?}
  \subsection{Changes between SSPL and GPL3}
  "The only substantive modification is section 13, which makes clear the condition to offering MongoDB as a service. A company that offers a publicly available MongoDB as a service must release the software it uses to offer such service under the terms of the SSPL, including the management software, user interfaces, application program interfaces, automation software, monitoring software, backup software, storage software and hosting software, all such that a user could run an instance of the service using the source code made available." \footnote{https://www.mongodb.com/licensing/server-side-public-license/faq}

  The SSPL is a different license that tries to solve the same problem as the Affero GPL, which says that it "is a modified version of the ordinary GNU GPL version 3. It has one added requirement: if you run a modified program on a server and let other users communicate with it there, your server must also allow them to download the source code corresponding to the modified version running there" \footnote{https://en.wikipedia.org/wiki/GNU\_Affero\_General\_Public\_License}. While the AGPL was made to expand the copyleft to programs running on a server, it did not solve the problem of cloud vendors hiding their entire infrastructure behind proprietary code. Effectively, they were scaling up and profiting from open source software without sharing the advancement they made with the community that offered the same profitable code. The SSPL blocks cloud giants from serving open source software since they are not willing to share how they created their infrastructure, while, theoretically, if someone else wanted to offer the same SaaS as the original developer it could do so by sharing the full architecture of its business.

  % * https://www.mongodb.com/licensing/server-side-public-license/faq
  % The only substantive modification is section 13, which makes clear the condition to offering MongoDB as a service. A company that offers a publicly available MongoDB as a service must release the software it uses to offer such service under the terms of the SSPL, including the management software, user interfaces, application program interfaces, automation software, monitoring software, backup software, storage software and hosting software, all such that a user could run an instance of the service using the source code made available.

  % * https://en.wikipedia.org/wiki/GNU_Affero_General_Public_License
  % is a modified version of the ordinary GNU GPL version 3. It has one added requirement: if you run a modified program on a server and let other users communicate with it there, your server must also allow them to download the source code corresponding to the modified version running there.
  
  \pagebreak
  \subsection{Why SSPL is not an open source license}
  The Open Source Foundation has deemed the SSPL to be not an open source one \footnote{https://opensource.org/sspl-not-open-source}. The SSPL takes rights away from the user, in violation of the Open Source Definition 6 "No Discrimination Against Fields of Endeavor" which states: "The license must not restrict anyone from making use of the program in a specific field of endeavor" \footnote{https://opensource.org/osd}. Furthermore, the OSF explains how this loss of rights is especially hard on those volunteers who contributed to the software and then cannot enjoy the fruits of their work anymore. In another article, the OSF explains the dangers of "fauxpen" licenses: "Most fauxpen source licenses plainly don't comply with the requirements of the open source definition. The source code may be available, but they are not open source" \footnote{https://opensource.com/article/19/4/fauxpen-source-bad-business}. The OSF then accuses the fauxpen source vendors to disrespect the open source principles and use the name as a business stratagem to attract clients.

  % * https://opensource.org/sspl-not-open-source
  % We've seen that several companies have abandoned their original dedication to the open source community by switching their core products from an open source license, one approved by the Open Source Initiative, to a "fauxpen” source license. The hallmark of a fauxpen source license is that those who made the switch claim that their product continues to remain "open” under the new license, but the new license actually has taken away user rights. The license du jour is the Server Side Public License. This license was submitted to the Open Source Initiative for approval but later withdrawn by the license steward when it became clear that the license would not be approved. Open source licenses are the foundation for the open source software ecosystem, a system that fosters and facilitates the collaborative development of software. Fauxpen source licenses allow a user to view the source code but do not allow other highly important rights protected by the Open Source Definition, such as the right to make use of the program for any field of endeavor. By design, and as explained by the most recent adopter, Elastic, in a post it unironically titled "Doubling Down on Open,” Elastic says that it now can "restrict cloud service providers from offering our software as a service” in violation of OSD6. Elastic didn't double down, it threw its cards in. And the software commons are now poorer for it. The Elastic projects were offered under the Apache license. Outside contributors donated time and energy with the understanding that their work was going towards the greater good, the public software commons. Now, instead, their contributions are embedded in a proprietary product. If they want to enjoy the fruits of their own and their co-contributors' labor, they have to agree to a proprietary license or fork.

  % * https://opensource.com/article/19/4/fauxpen-source-bad-business
  % Proprietary software masquerading as open source —what has been termed fauxpen source— is toxic. It's an intentionally deceptive hybrid that seeks to give its proponents the best of both worlds—the positive vibe and broad distribution of open source, together with the commercial leverage of proprietary software.
  % What does it mean to be "open source"? It doesn't just mean that the source code is available—it means a lot more. To keep everyone on the same page, the nonprofit Open Source Initiative was formed in 1998 to define and govern the use of the term "open source." Most fauxpen source licenses plainly don't comply with the requirements of the open source definition. The source code may be available, but they are not open source.
  % Proponents of fauxpen source frequently offer the defense that they are protecting themselves from free riders who would use their software without subsidizing its creation and maintenance. Meanwhile, those same fauxpen source vendors freely appropriate millions of lines of code from other open source projects to power their own businesses without contributing back anything at all to most of those projects. In reality, fauxpen source vendors themselves are free-riding on the efforts of the open source community, but in an even more pernicious way. Instead of adding to the commons they rely on, they are appropriating the open source brand while withholding their own software from that commons. That's an unfair trade.

  Other free software institutions also deemed problematic this new license in contrast with the principles that promote sharing of information and freedom to access programs. Fedora spokesman Tom Callaway tells in the Fedora Forum how "the SSPL is intentionally crafted to be aggressively discriminatory towards a specific class of users. Additionally, it seems clear that the intent of the license author is to cause Fear, Uncertainty, and Doubt towards commercial users of software under that license" 
  \footnote{https://lists.fedoraproject.org/archives/list/devel@lists.fedoraproject.org/thread/\\IQIOBOGWJ247JGKX2WD6N27TZNZZNM6C/}. 
  
  "The SSPL is clearly not in the spirit of the DFSG (Debian's free software guidelines), let alone complimentary to the Debian's goals of promoting software or user freedom", mentioned Chirs Lamb, Debian Project Leader 
  \footnote{https://hub.packtpub.com/red-hat-drops-mongodb-over-concerns-related-to-its-\\server-side-public-license-sspl/}. 
  
  Richard Fontana, senior commercial counsel at Red Hat Inc., tells "I have some concerns over what cloud providers are doing, but my bigger concern is how [Redis, Confluent and MongoDB] are reacting.” Fontana said he's particularly concerned about contract language that uses open source terminology to define licenses that are really proprietary. 
  \footnote{https://siliconangle.com/2018/12/22/stormy-weather-open-source-software-firms-\\limit-licenses-stop-cloud-giants/}

  % * https://lists.fedoraproject.org/archives/list/devel@lists.fedoraproject.org/thread/IQIOBOGWJ247JGKX2WD6N27TZNZZNM6C/
  % It is the belief of Fedora that the SSPL is intentionally crafted to be aggressively discriminatory towards a specific class of users. Additionally, it seems clear that the intent of the license author is to cause Fear, Uncertainty, and Doubt towards commercial users of software under that license. To consider the SSPL to be "Free" or "Open Source" causes that shadow to be cast across all other licenses in the FOSS ecosystem, even though none of them carry that risk.

  % * https://hub.packtpub.com/red-hat-drops-mongodb-over-concerns-related-to-its-server-side-public-license-sspl/
  % Tom Callaway, University outreach Team lead, Red Hat, mentioned in a note, earlier this week, that Fedora does not consider MongoDB's Server Side Public License v1 (SSPL) as a Free Software License. He further explained that SSPL is "intentionally crafted to be aggressively discriminatory towards a specific class of users. To consider the SSPL to be "Free” or "Open Source” causes that shadow to be cast across all other licenses in the FOSS ecosystem, even though none of them carry that risk”.
  % The first instance of Red Hat removing MongoDB happened back in November 2018 when its RHEL 8.0 beta was released. RHEL 8.0 beta release notes explicitly mentioned that the reason behind the removal of MongoDB in RHEL 8.0 beta is because of SSPL.
  % Apart from Red Hat, Debian also dropped MongoDB from its Debian archive last month due to similar concerns over MongoDB's SSPL. "For clarity, we will not consider any other version of the SSPL beyond version one. The SSPL is clearly not in the spirit of the DFSG (Debian's free software guidelines), let alone complimentary to the Debian's goals of promoting software or user freedom”, mentioned Chirs Lamb, Debian Project Leader.

  % * https://siliconangle.com/2018/12/22/stormy-weather-open-source-software-firms-limit-licenses-stop-cloud-giants/
  % "I'm worried about anything that blurs the boundaries around what the community has defined as the scope of open source,” said Richard Fontana, senior commercial counsel at Red Hat Inc. "I have some concerns over what cloud providers are doing, but my bigger concern is how [Redis, Confluent and MongoDB] are reacting.” Fontana said he's particularly concerned about contract language that uses open source terminology to define licenses that are really proprietary. For example, "commons” is a term that connotes resources that are available to all, such as air and water. But he said the Commons Clause adopted by Redis "is really not a commons in the way we understand it. It's intended to close the software by tacking a restriction onto a standard open source license. I worry that's going to confuse developers.”
  \pagebreak
  \subsection{"Free and Open" products}
  It should be noted that the open source institutions mentioned above do not suffer from cloud vendors as much as the businesses adopting SSPL. Those companies still consider themselves to be part of the community that shares the same principles of free software, freedom of usage and leaving the source code open for everyone to look at and modify: while they have written in their blogs that they should not be named as Open Source, they now refer to themselves as free and open: "While we have chosen to avoid confusion by not using the term open source to refer to these products, we will continue to use the word "Open” and "Free and Open.” These are simple ways to describe the fact that the product is free to use, the source code is available, and also applies to our open and collaborative engagement model in GitHub. We remain committed to the principles of open source - transparency, collaboration, and community" writes Elastic in his FAQ explaining the change of license \footnote{https://www.elastic.co/pricing/faq/licensing}.

  % * https://www.elastic.co/pricing/faq/licensing
  % Neither the Elastic License nor SSPL have been approved by the OSI, so to prevent confusion, we no longer refer to Elasticsearch or Kibana as open source. We updated our website and our messaging to refer to these products as "Free and Open,” and when talking about the licenses directly, we describe them as "source-available.” If you notice an area we missed, please let us know, so we can correct it.
  % While we have chosen to avoid confusion by not using the term open source to refer to these products, we will continue to use the word "Open” and "Free and Open.” These are simple ways to describe the fact that the product is free to use, the source code is available, and also applies to our open and collaborative engagement model in GitHub. We remain committed to the principles of open source - transparency, collaboration, and community.

  Heather Meeker is one of the world's foremost legal experts on open source software licensing and compliance. She's authored the go-to book on the topic, Open Source for Business, and is a General Partner at OSS Capital \footnote{https://fossa.com/blog/q-a-heather-meeker-hot-topics-oss-license-compliance/}. She's been the main helper in creating the SSPL, the RSALv2 and the Elastic License \footnote{https://redis.com/blog/rsalv2-sspl-announcement/}. In the Commercial Open Source Software (COSS) community, she writes about open source and what these new licenses mean. In her post "SSPL Re-Takes the Stage in 2021" \footnote{https://www.coss.community/cossc/sspl-re-takes-the-stage-in-2021-2koa} she criticizes the OSF for its lack of transparency on the criteria of whether a license is open source or not. To justify the open-sourceness of the SSPL she starts with the "Freedom Zero”: "The freedom to run the program as you wish, for any purpose" \footnote{https://www.gnu.org/philosophy/free-sw.en.html}. SSPL was the first license trying to be open source while limiting the power of cloud companies, unlike the other source available licenses - licenses with restrictions that do not meet the OSD, like the Confluent Community one. The discussion born from the MongoDB license created such a fierce debate that the request to the OSF was removed before a final verdict (whether SSPL is open source or not). In particular, the arguments against the SSPL were mostly against its creators, which were deemed looking for profit more than driven by a sense of morality. Another problem is the lack of definition of what "guaranteeing software freedom" actually means since the same case used for SSPL could be used against the AGPL.

  % * https://fossa.com/blog/q-a-heather-meeker-hot-topics-oss-license-compliance/
  % Heather Meeker is one of the world's foremost legal experts on open source software licensing and compliance. She's authored the go-to book on the topic, Open Source for Business, and is a General Partner at OSS Capital.

  % * https://redis.com/blog/rsalv2-sspl-announcement/
  % We worked closely on the updated RSALv2 with Heather Meeker, who is well known for helping to draft many OSS licenses, including the Mozilla Public License 2.0 and source-available licenses like the Confluent Community License, SSPL, Elastic License 2.0, and others. We hope this change clarifies our intent and addresses the questions we've received about the RSALv1 license over the past few years.

  % * https://www.coss.community/cossc/sspl-re-takes-the-stage-in-2021-2koa
  % "Freedom Zero”: "The freedom to run the program as you wish, for any purpose." https://www.gnu.org/philosophy/free-sw.en.html
  % Third, and most important, OSI approval is a sufficient but not a necessary condition for a license to meet the open source definition. Once upon a time, OSI approved licenses based on whether they fit the OSD. But today, OSI also imposes other criteria to approve a license. OSI's website cites these rules for the approval process (https://opensource.org/approval). To understand the bid and ask on this topic, you have to understand the process of OSI approval and the facts behind SSPL. SSPL was created by MongoDB to address the so-called "AWS stripimininig problem.” At the time, many companies were adopting source available licenses -- licenses with restrictions that do not meet the OSD, but MongoDB took a different approach. SSPL was applied to the MongoDB software, and submitted by MongoDB for approval to OSI. After months of discussion, it was withdrawn from consideration for OSI approval. SSPL was never approved or rejected by OSI. Adding to the confusion was the facts that OSI offers no definition of what "guarantees software freedom." On a certain level, no license -- Apache or BSD or GPL or AGPL -- can do this. In fact, the motivation for creating a license like SSPL was a view that AGPL was not sufficient to compel all users to share their changes. 
  
  % \subsection{Public Opinions on SSPL}
  % * https://www.percona.com/blog/2020/06/16/why-is-mongodbs-sspl-bad-for-you/
  % MongoDB has always been a "reluctant open source company”. While the world was moving away from copy-left licenses (GPL) to permissive ones (MIT, BSD, Apache), MongoDB chose AGPL, an even more restrictive version of the GPL license for their MongoDB Server Software. If you read MongoDB's S1 form used for their IPO filing you will see the emphasis is on the "freemium” model. This is achieved by crippling the Community Server version, rather than giving support to Open Source community values. In his 2019 interview, MongoDB CEO Dev Ittycheria confirms that MongoDB Inc. does not care about working with the open source community to make MongoDB better, as their focus is on their freemium strategy: "MongoDB was built by MongoDB. There was no prior art. We didn't open source it for help; we open-sourced it as a freemium strategy,” Dev Ittycheria, MongoDB CEO. 
  % SSPL requires anyone who wants to offer MongoDB as a DBaaS to either release all surrounding infrastructure as SSPL or get a commercial license from MongoDB. The first is impractical for cloud vendors as licensing MongoDB directly allows MongoDB Inc. to exercise significant control over end-user pricing, meaning there is no true competition. With DBaaS becoming the leading form of database software consumption, this vendor lock-in is quite an issue!
  % * https://lists.fedoraproject.org/archives/list/devel@lists.fedoraproject.org/thread/IQIOBOGWJ247JGKX2WD6N27TZNZZNM6C/
  % Mat K. Witts: "[...the SSPL is intentionally crafted to be aggressively discriminatory towards a specific class of users]. Who are they? Can you identify this class of users please? Thanks." 
  % Peter Robinson: "You can find the details quite easily using google if you so desire." 
  % Mat K. Witts: "wow... are you really too embarrassed to admit it is basically Amazon and other cloud-based monsters? good... you ought to be - lol..."
  % * https://www.reddit.com/r/freesoftware/comments/l058et/what_are_your_thoughts_on_the_sspl/
  % u/agsking: "I'm conflicted. On the one hand, I think free as in freedom software is a good thing. On the other hand, I think the people who develop free software are being [abused] over by companies like Amazon. I also think that expecting companies/individuals to voluntarily donate an amount proportional to their benefit from software is unrealistic. I don't have any good solutions, but I'm interested in hearing discussion around this."
  % * https://www.reddit.com/r/opensource/comments/9onv21/fed_up_with_cloud_giants_ripping_off_its_database/
  % u/undu: "According to this only Mongo is allowed to offer it as a service with private bits, even if other player contribute to the source code of MongoDB, or as they say "contribute to the community" Meh, it's clear this has nothing to do with "contributing to the community", this is a decision purely made to get return on investment."
  % u/madpew: "So basically: Open source project fed up with others making profit re-licensing to force them into license-negotiation so they can make profit."
  % u/masterdirk: "Nothing inherently wrong with wanting to make a profit. In my opinion it's one of the worst document-databases, but we still use it a lot."
  % * https://www.reddit.com/r/programming/comments/ah74dw/mongodb_opensource_server_side_public_license/
  %u/RaptorXP: "If cloud giants couldn't use Linux, Linux would be a lot less interesting."
  % u/FluorineWizard: "I think the operative difference here is that Linux has benefitted plenty from the corporate world giving back over time, even if most users don't give back. An operating system also indirectly benefits from all users because building an ecosystem encourages other people to adopt the platform, who might eventually contribute.
  % SaaS and cloud providers are a pretty different beast. Hence the multiple (if unsuccessful/unpopular) attempts at getting the big guys to play ball by restricting the license. This kind of problem isn't new. I personally dislike the GPL but I find it entirely appropriate when applied to complete, end-user oriented software, to prevent predatory rebranding. There's a reason why projects like Paint.NET went back to closed-source freeware.
  % I frankly couldn't tell you if MongoDB are doing it right or how they could do better, but I can absolutely see where they're coming from. There's just no convenient, enforceable "don't be a [bad person]" license that covers its bases in today's context without being a PITA."

  \pagebreak
  \section{Personal Considerations on the subject}
  \subsection{The goals of Free Software, and why SSPL respects them}
  The core of the Free Software ideology is that the user should have full control of the programs that run on their machines. Free, as in free speech, means that people should see the source code of the program to understand what it does, and to be sure that it does not hide trackers or malicious code. Opening the source code means that people can run and modify the software so that it can be improved. Nonexpert users know from the community that the program is safe, and there will always be a volunteer who will make the experience better. The success of this ideology stands in the fact that knowledge and ideas are shared, without bureaucracy and legal blocks, so that anyone can build on top of other people's code to advance the entire digital ecosystem. Richard Stallman, the creator of the GNU system, talks about this in a TedTalk available on YouTube called "Free software, free society: Richard Stallman at TEDxGeneva 2014" \footnote{https://youtu.be/Ag1AKIl\_2GM}.

  For the vast majority of users, the SSPL does not remove this power from the software licensed with it. Its goals remain the same as the free software community, and the SSPL was created with the specific objective of blocking monolitic corporations that have the power to hold for themselves what the community created together. MongoDB could have just gone the same way as other companies did, restricting a user's possibility of usage of their code, but instead, they have tried a way to save the open source community by obligating the cloud giants to participate in the open source environment, a thing they have always be reluctant. The license is by no mean perfect and it surely needs corrections, but the best place where this modification can happen is the same one that allows coding as we know it today a much easier task, the open source community. 

  \pagebreak
  \subsection{Why we should protect Open Source businesses from Cloud Companies}
  While the vast majority of open source software is small projects run by volunteers, some ideas need much more than a couple of hours each week. Complex projects, like big-scale databases or operating systems, require people to work on it full-time jobs, but they cannot do that if they are not able to earn money for food or rent at the end of the day. Confluent CEO Jay Kreps says, in the blog post explaining its license change: "There are thousands of libraries thriving on GitHub that don't need much investment beyond a few volunteer contributors. Distributed data systems are different. Building a successful new distributed data platform is just excruciatingly hard. [...] What I think is most stark is that the only systems that remained relevant through to today are those that, whatever their origin, managed to develop a stable commercial entity that helped sustain ongoing investment. Those that did this (MongoDB, ElasticSearch, Cassandra, Hadoop) all continue to thrive and have become part of the modern stack. Those that didn't (Voldemort, Dynomite, CouchDB, and a dozen others) have all fallen by the wayside, despite early popularity. They still exist, but most likely you have never heard of them". \footnote{https://www.confluent.io/blog/license-changes-confluent-platform/}
  
  There are a lot of ways an open-source business can make money to pay its developers: for example, getting acquired by a big tech company, like Red Hat recently acquired by IBM \footnote{https://www.redhat.com/en/ibm}. But some types of software can only make money by offering their software as a service itself, like the ones we talked about here, and unfair competition that only steals and not contributes means only that products evolve slower for the greed of an external player who can exploit the morality of benevolent people. Open Source business making money is good because it means the best of both worlds: we have a strong company working actively on a product while having the security and the advantages of the open source community watching and suggesting changes on the software itself. But if we allow them to be exploited by bigger companies that gain a lot from this practice, we are telling the open source community that it is not feasible to make money from this type of software, pushing towards closed license programs.

  \pagebreak
  \subsection{Tangible problems of the power of Amazon}
  In the previous paragraphs, it was preferred a neutral stance on the subject, calling against the general "Cloud Vendors". The practical truth, however, is that the main culprit is the Amazon Web Service. Each one of the businesses we talked about fighting against AWS offering a competitive service without contributing back, and when confronted, instead of starting a fruitful collaboration, it always decided to play dirty, forking the project and then offering a compatible version (like DocumentDB or OpenSearch, the forked version of ElasticSearch). Considering that all software using SSPL has still open source code, it usually means that AWS copies the new features and stays on top, without working on them.

  Of course, we need to be wary of other cloud vendors, but most open source businesses already have economic ties with them, like Confluent and its partnership with Azure \footnote{https://www.confluent.io/partner/microsoft-azure/}, or Elastic: "We have built strong relationships with Google Cloud, Microsoft Azure, Alibaba Cloud, and Tencent Cloud" \footnote{https://www.elastic.co/pricing/faq/licensing}
  
  This may look like a war for profit, where two competing tech companies want to have as much money as they want, but the bad practices and the monopoly of AWS have real, tangible consequences in the real world, in fields far from tech and software. 

  In its excellent video "Is Amazon too big to fail?" \footnote{https://youtu.be/EYPs-ya\_GDA}, the YouTuber and disseminator PolyMatter explains how Amazon makes so much money thanks to AWS (nearly 60\% of the total income in 2018) that it can allow the main company to compete in virtually any sector (online retail store, video and music streaming, ebooks, physical stores, and many others), at a massive scale and without the need for profit. Most of the time, it allows its other subsidiaries to stay in red to acquire an unfair monopoly in sectors where other competitors, that need to make a profit, cannot compete on the same level.

  It is no surprise then how Amazon Shopping can keep the price so low, offering things like free shipping and fast delivery, while the competitors cannot: it does not make a profit from it. And this has big effects on the entire population: local shops must close because they cannot stay afloat, while postmen and warehouse workers suffer inhuman conditions and low pay. 

  \pagebreak
  Stories can be quickly found online, and it happens anywhere Amazon operates: in the USA \footnote{https://www.theguardian.com/technology/2020/feb/05/amazon-workers-protest-\\unsafe-grueling-conditions-warehouse}, in Italy \footnote{https://www.ilpost.it/2021/10/22/sfruttamento-logistica-emilia-romagna/} and in Mexico \footnote{https://www.vice.com/en/article/pkb9qn/amazon-to-open-dollar21-million-state-of-\\the-art-warehouse-in-tijuana-slum}, among many other contries.
  
  This cycle of injustice is possible because Amazon has a large power on cloud infrastructure, a power that it got thanks to all the services it can provide on AWS, where they are integrated all together and it is possible to combine them, for a developer, with just some clicks. And this aggregation of options is made at the expense of open source software creators. Fighting against corporations, via more restrictive licenses, is much more than just considering the user right of offering a SaaS removed: if the original creator cannot compete with cloud vendors, who can?
\end{document}
